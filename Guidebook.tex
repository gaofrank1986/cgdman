\documentclass[oneside,12pt]{cgd_book}

\reversemarginpar
\parindent=0pt
\parskip=.8\baselineskip

\begin{document}

%%%%%%------------------cover--------------------------
\begin{titlepage}
\raggedleft
\null

\vskip 4em
\SectionFont{\fontsize{40}{80}\selectfont  Genius}

\bigskip

{\fontsize{25}{80}\selectfont  Semiconductor Device Simulator}

\bigskip

{\Large Version 1.7.4}

\bigskip

{\fontsize{25}{80}\selectfont  Genius User's Guide}


\vfill


\centerline{Cgenda Pte Ltd}


\end{titlepage}


\thispagestyle{empty}
Copyright (c) 2008-2009 Cogenda Pte Ltd, Singapore.

All rights reserved.
\par
\marginhead{License Grant}Duplication of this documentation is permitted only for internal use within the organization of the
        licensee.
\par
\par
\marginhead{Disclaimer}THIS DOCUMENTATION IS PROVIDED BY THE COPYRIGHT HOLDERS AND CONTRIBUTORS "AS IS" AND ANY EXPRESS OR
        IMPLIED WARRANTIES, INCLUDING, BUT NOT LIMITED TO, THE IMPLIED WARR\-A\-N\-TIES OF MERCHANTABILITY AND FITNESS FOR A
        PARTICULAR PURPOSE ARE DISCLA\-I\-MED. IN NO EVENT SHALL THE COPYRIGHT OWNER OR CONTRIBUTORS BE LIABLE FOR ANY
        DIRECT, INDIRECT, INCIDENTAL, SPECIAL, EXEMPLARY, OR CONSEQUENTIAL DAMAGES (INC\-L\-U\-DING, BUT NOT LIMITED TO,
        PROCUREMENT OF SUBSTITUTE GOODS OR SERVICES; LOSS OF USE, DATA, OR PROFITS; OR BUSINESS INTERRUPTION) HOWEVER
        CAUSED AND ON ANY THEORY OF LIABILITY, WHETHER IN CONTRACT, STRICT LIABILITY, OR TORT (INCLUDING NEGLIGENCE OR
        OTHERWISE) ARISING IN ANY WAY OUT OF THE USE OF THIS SOFTWARE, EVEN IF ADVISED OF THE POSSIBILITY OF SUCH
        DAMAGE.

\newpage




\tableofcontents

\pagestyle{heading}
\chapter{Physics in Genius Device Simulator}
Since Gummel's original work, the drift-diffusion model has been widely used in the semiconductor device
    simulation. It is now the de facto industry standard in this field.
\par
The original DD model can be achieved by following approximation from hydrodynamic model:
\begin{compactitem}
\item Light speed is much faster than carrier speed.
\item All the collision is elastic.
\item Bandgap does not change during collision.
\item Carrier temperature equals to lattice temperature and keeps equilibrium.
\item The gradient of driving force should keep small.
\item Carrier degenerate can be neglected.
\end{compactitem}

Some improvements have been applied to DD model for extend its capability.
These ``patches'' of course make things complex, but they can deal with real problems.



This chapter describes the DD model and its variations used by GENIUS code for
describing semiconductor device behavior as well as physical based parameters
such as mobility, recombination rate and son on.


Some improvements have been applied to DD model for extend its capability. These "patches" of course make
    things complex, but they can deal with real problems.
\begin{equation}
\nabla \cdot \varepsilon \nabla \psi = - q\left( p - n + {N_D}^+ - {N_A}^- \right)
\end{equation}
where, $\psi$ is the electrostatic potential of the vacuum level. This
      choice makes the description of metal-oxide-semiconductor contact and heterojunction easier.
$n$ and $p$ are the electron and hole concentration, ${N_D}^{+}$
and ${N_A}^{-}$ are the ionized impurity
      concentrations. $q$ is the magnitude of the charge of an electron.
\par
The relationship of conduct band $E_c$, valence band $E_v$ and vacuum level
$\psi$ is:
\begin{align}
  E_c  &=-q\psi-\chi-\Delta E_c \\
  E_v   &=E_c-E_g+\Delta E_v.
\end{align}

\section{Level 1 Drift-Diffusion Equation}
Level 1 Drift-Diffusion ( DDML1 ) is the fundamental solver of GENIUS code for
lattice temperature keeps constant though out the solve procedure.

\marginhead{subjection}The primary function of DDML1 is to solve the following set of partial differential
equations, namely Poisson's equation, along with the hole and electron continuity
equations:

\section{Level 1 Drift-Diffusion Equation}
Level 1 Drift-Diffusion (\cal{DDML1})  is the fundamental solver of GENIUS code for lattice temperature keeps constant though out the solve
      procedure.
\par
The primary function of \cal{DDML1} is to solve the following set of partial differential
      equations, namely Poisson's equation, along with the hole and electron continuity equations:
\subsection{Poisson's Equation}
\begin{equation}
\nabla \cdot \varepsilon \nabla \psi = - q\left( p - n + {N_D}^+ - {N_A}^- \right)
\end{equation}\par\par
\label{Poisson's Equation}where, $\psi$ is the electrostatic potential of the vacuum level. This
      choice makes the description of metal-oxide-semiconductor contact and heterojunction easier.
$n$ and $p$ are the electron and hole concentration, ${N_D}^{+}$
and ${N_A}^{-}$ are the ionized impurity
      concentrations. $q$ is the magnitude of the charge of an electron.
\par
The relationship of conduct band $E_c$, valence band $E_v$ and vacuum level
$\psi$ is:
\begin{subequations}
\begin{align}
 E_c  &=-q\psi-\chi-\Delta E_c\\
 E_v  &=E_c-E_g+\Delta E_v.
\end{align}
\end{subequations}
\par
Here, $\chi$ is the electron affinity. $E_g$ is the
      bandgap of semiconductor. $\Delta E_c$ and $\Delta E_v$ are the bandgap shift caused by heavy doping or mechanical strain.
\par
Furthermore, the relationship of vacuum level $\psi$ and intrinsic Fermi
      potential $\psi_ {\rm intrinsic}$ is:
\begin{equation}
\psi = \psi _{\rm intrinsic} - \frac{\chi}{q} - \frac{E_g}{2q} - \frac{k_b T}{2q}\ln \left(
          \frac{N_c}{N_v} \right)
\end{equation}
\par
The reference $0\eVolt$ of energy is set to intrinsic Fermi level of equilibrium
      state in GENIUS.
\par
\subsection{Continuity Equations}\label{Continuity equation+DDML1}The continuity equations for electrons and holes are defined as follows:
\begin{subequations}
\begin{align}
 \frac{\partial n}{\partial t}  &= \frac{1}{q}\nabla \cdot \vec{J}_n - (U - G)\\
 \frac{\partial p}{\partial t}  &= - \frac{1}{q}\nabla \cdot \vec{J}_p - (U - G)
\end{align}
\end{subequations}
where $\vec{J}_n$ and $\vec{J}_p$ are the electron
      and hole current densities, $U$ and $G$ are the
      recombination and generation rates for both electrons and holes.
\par
\subsection{Drift-Diffusion Current Equations}\label{Drift-diffusion current+DDML1}The current densities
$\vec{J}_n$ and $\vec{J}_p$ are expressed in terms of the level 1 drift-diffusion model here.
\begin{subequations}
\begin{align}
 \vec{J}_n  &= q\mu_n n \vec{E}_n + q D_n \nabla n\\
 \vec{J}_p & = q\mu_p p \vec{E}_p - q D_p \nabla p
\end{align}
\end{subequations}
where $\mu_n$ and $\mu_p$ are the
      electron and hole mobilities. $D_n=\frac{k_bT}{q}\mu_n$ and
$D_p=\frac{k_bT}{q}\mu_p$ are the electron and hole diffusivities, according to Einstein
      relationship.
\par
\subsection{Effective Electrical Field}$\vec{E}_n$ and $\vec{E}_p$
are the effective
        driving electrical field to electrons and holes, which related to local band diagram. The band structure of
        heterojunction has been taken into account here \cite{Lindefelt1994}.
\begin{subequations}
\label{eq:Equation:DDML1:DrivingField}
\begin{align}
 \vec{E}_n  &= \frac{1}{q}\nabla E_c - \frac{k_b T}{q}\nabla \left( \ln (N_c ) - \ln (T^{3/2} )
        \right)\\
 \vec{E}_p  &= \frac{1}{q}\nabla E_v + \frac{k_b T}{q}\nabla \left( \ln (N_v ) - \ln (T^{3/2} )
        \right)
\end{align}
\end{subequations}
The lattice temperature keeps uniform throughout \cal{DDML1}, the above temperature
      gradient item takes no effect in fact.
\par
By substituting drift-diffusion model into the current density expressions, and combining with Poisson's
      equation, the following basic equations for \cal{DDML1}
are obtained:
\begin{subequations}
\begin{align}
 \frac{\partial n}{\partial t}  &= \nabla \cdot \left (\mu_n n \vec{E}_n + \mu_n\frac{k_b T}{q}\nabla n \right
        ) - (U - G)\\
 \frac{\partial p}{\partial t}  &= -\nabla \cdot \left (\mu_p p\vec{E}_p - \mu_p\frac{k_b T}{q}\nabla p \right
        ) - (U - G)\\
 \nabla \cdot \varepsilon \nabla \psi  &= - q(p - n + {N_D}^{+} - {N_A}^{-} )
\end{align}
\end{subequations}
\cal{DDML1} is suitable for PN diode, BJT transistor and long gate MOSFET simulation. It is
      robust, and runs pretty fast for real work. The detailed discretization scheme can be found at [[TODO]].
\par
\section[sec:Equation:DDML2]{Level 2 Drift-Diffusion Equation}
\label{DDML2+Equation}The Level 2 DD model considers the influence of lattice temperature by solving the extra thermal
      equation simultaneously with the electrical equations. Also, the formula of drift-diffusion equation should be
      modified according to \cite{Selberherr1984}.
\par
\label{Drift-diffusion current+DDML2}The electron diffusion current in
\cal{DDML1} can be written as:
\par

\begin{equation}
{\vec{J}}_{n,{\rm diff}} = \frac{k_b T}{q}\mu_n q\nabla n = k_b T\mu_n \nabla n
\end{equation}
\subsection{Temperature Gradient Correction}But for DDML2, it has the form of
\begin{equation}
{\vec{J}}_{n,{\rm diff}} = \mu_n k_b (T\nabla n + n\nabla T)
\end{equation}
The hole diffusion current should be modified in the same manner.
\begin{equation}
{\vec{J}}_{p,{\rm diff}} = -\mu_p k_b (T\nabla p + p\nabla T)
\end{equation}
\subsection{Heat Flow Equation}\label{Heat flow equation+DDML2}The following heat flow equation is used:
\begin{equation}
\rho c_p \frac{\partial T}{\partial t} = \nabla \cdot \kappa \nabla T + \vec{J} \cdot \vec{E} + (E_g + 3k_b T) \cdot
        (U - G)
\end{equation}
where $\rho$ is the mass density of semiconductor material. $c_p$
is the heat capacity. $\kappa$ is the thermal conductivity
      of the material. $\vec{J}\cdot\vec{E}$ is the joule heating of current.
$(E_g + 3k_b T) \cdot (U - G)$ is lattice heating due to carrier recombination and
      generation.
\par
From above discussion, the governing equations for DDML2 are as follows:
\begin{subequations}
\begin{align}
 \frac{\partial n}{\partial t}  &= \nabla \cdot \left(\mu_n n \vec{E}_n + \mu_n \frac{k_b T}{q}\nabla n + \mu_n
        \frac{k_b \nabla T}{q} n\right) - \left( {U - G} \right)\\
 \frac{\partial p}{\partial t}  &= -\nabla \cdot \left(\mu_p p \vec{E}_p - \mu_p \frac{k_b T}{q}\nabla p -
        \mu_p \frac{k_b \nabla T}{q} p\right) - \left( {U - G} \right)\\
 \nabla \cdot \varepsilon \nabla \psi  &= - q\left( {p - n + {N_D}^+ - {N_A}^ - } \right)\\
 \rho c_p \frac{\partial T}{\partial t}  &= \nabla \cdot \kappa \nabla T + \vec{J} \cdot \vec{E} + (E_g + 3k_b
        T) \cdot (U - G)
\end{align}
\end{subequations}
This model can be used as power transistor simulation as well as breakdown simulation. Unfortunately, nearly
      all the physical parameters are related with temperature. They should be considered during self consistent
      simulation, which greatly slows down the speed. The \cal{DDML2}
solver runs $50-70\Percent$ slower than \cal{DDML1}. However, it seems no convergence
      degradation happens in most of the case. The discretization scheme can be found at [[TODO]].
\par
\section[sec:Equation:EBML3]{Level 3 Energy Balance Equation}
\label{EBML3+Equation}Energy Balance Model \cite{PISCES-2ET}
is introduced into GENIUS code for simulating
      short channel MOSFET. This is a simplification of full hydrodynamic (HD) model
\cite{Aste2003}. The
      current density expressions from the drift-diffusion model are modified to include additional coupling to the
      carrier temperature. Also, reduced carrier energy conservation equations, which derived from second order moment
      of Boltzmann Transport Equation, are solved consistently with drift-diffusion model. The simplification from HD to
      EB makes sophisticated Scharfetter-Gummel discretization still can be used in the numerical solution, which
      ensures the stability.
\par
\subsection{Current Equation for EBM}\label{Drift-diffusion current+EBML3}The current density
$\vec{J}_n$ and $\vec{J}_p$ are then expressed as:
\begin{subequations}
\begin{align}
 \vec{J}_n & = q\mu_n n \vec{E}_n + k_b \mu_n \left( {n\nabla T_n + T_n \nabla n} \right)\\
 \vec{J}_p  &= q\mu_p p \vec{E}_p - k_b \mu_p \left( {p\nabla T_p + T_p \nabla p} \right)
\end{align}
\end{subequations}
where, \textdollar{}T\_n\textdollar{} and \textdollar{}T\_p\textdollar{} are electron and hole temperature, respectively. The difference between above
      equations and carrier density equations in \cal{DDML2}
is lattice temperature replaced by carrier
      temperature.
\par
\subsection{Energy Balance Equations}\label{Energy-balance equation+EBML3}In addition, the energy balance model includes the following electron and hole energy balance
        equations:
\begin{subequations}
\begin{align}
 &\frac{\partial \left( {n\omega _n } \right)} {\partial t} + \nabla \cdot \vec{S}_n  = \vec{E}_n \cdot
        \vec{J}_n +H_n\\
 &\frac{\partial \left( {p\omega _p } \right)} {\partial t} + \nabla \cdot \vec{S}_p = \vec{E}_p \cdot
        \vec{J}_p +H_p
\end{align}
\end{subequations}
where, $\omega _n$ and $\omega _p$ are electron and
      hole energy. For HD model, the carrier energy includes thermal and kinetic terms
$\omega _c =
      \frac{3}{2}k_bT_c + \frac{1}{2}m^* v_c^2$, but only thermal energy for EB model
$\omega _c = \frac{3}{2}k_bT_c$. Here $c$ stands for $n$ or $p$.
$\omega_0=\frac{3}{2}k_bT$ is the carrier equilibrium energy, for carrier temperature equals to lattice temperature.
\par
$\vec{S}_n$ and $\vec{S}_p$ are the flux of
      energy:
\begin{subequations}
\begin{align}
 \vec{S}_n & = - \kappa _n \nabla T_n - \left( \omega_n + k_b T_n \right) \frac{\vec{J}_n} {q}\\
 \vec{S}_p & = - \kappa _p \nabla T_p + \left( \omega_p + k_b T_p \right) \frac{\vec{J}_p} {q}
\end{align}
\end{subequations}
The heat conductivity parameter for carriers can be expressed as:
\begin{equation}
\kappa_c=(\frac{2}{5}+\gamma)\frac{{k_b}^2}{q}T_c\mu_cc
\end{equation}
where $c$ stands for $n$ and $p$, respectively. The constant parameter
$\gamma$ equals $-0.7$ in the GENIUS software.
\par
The $H_n$ and $H_p$ are the rate of net loss of
      carrier kinetic energy:
\begin{subequations}
\begin{align}
 H_n &=  \left( R_{\rm Aug} - G \right) \cdot \left( E_g + \frac{3k_b T_p } {2} \right) -
        \frac{3k_b T_n } {2}\left( R_{\rm SHR} + R_{\rm dir} - G \right) \nonumber\\
  &\quad-\frac{{n\left( {\omega _n - \omega _0 } \right)}}{{\tau _n }}\\
 H_p &=  \left( R_{\rm Aug} - G \right) \cdot \left( E_g + \frac{3k_b T_n } {2} \right) -
        \frac{3k_b T_p}{2}\left( R_{\rm SHR} + R_{\rm dir} - G \right) \nonumber\\
 &\quad- \frac{p\left( \omega _p - \omega _0 \right)}{\tau _p }
\end{align}
\end{subequations}
\subsection{Lattice Heat Equation for EBM}\label{Heat flow equation+EBML3}At last, the lattice heat flow equation should be rewritten as:
\begin{equation}
\rho c_p \frac{\partial T}{\partial t} = \nabla \cdot \kappa \nabla T + H
\end{equation}
where
\begin{equation}
H = R_{\rm SHR} \cdot \left( E_g + \frac{3k_b T_p } {2} + \frac{3k_b T_n } {2} \right) + \frac{
        n\left( \omega _n - \omega _0 \right)} {\tau _n } + \frac{p\left( \omega _p - \omega _0 \right)}{\tau _p
        }
\end{equation}
The carrier energy is mainly contributed by joule heating term $\vec{E}_c\cdot
      \vec{J}_c$, and heating (cooling) due to carrier generation (recombination) term. The carriers exchange
      energy with lattice by collision, which described by energy relaxation term
$\tau_{\omega
      _c}$. This model is suitable for sub-micron MOS (channel length
$1\sim 0.1
      \uMeter$) and advanced BJT simulation. However, the computation burden of EB method is much higher than
      DD. And the convergence of EB solver is difficult to achieve, which requires more strict initial value and more
      powerful inner linear solver. The discretization scheme can be found at [[TODO]].
\par
From above discussion, all the governing equations of DD/EB method is elliptical or parabolic. From
      mathematic point of view, does not like hyperbolic system\footnote{
One have to face discontinuous problem, i.e. shock wave.
\par
}, the solution of elliptical or parabolic system is always smooth. The required numerical technique
      is simple and mature for these systems. As a result, the DD and EB method is preferred against full hydrodynamic
      method.
\par
\section{Band Structure Model}
The band structure parameters, including bandgap $E_g$, effective density of
      states in the conduction band $N_c$ and valence band $N_v$, and intrinsic carrier concentration
$n_{ie}$, are the most
      important and fundamental physical parameters for semiconductor material
\cite{Sze1981}.
\par
\subsection{Effective Density of States}Effective density of states\label{Density of states+effective}
in the conduction and valence band are defined as follows:
\begin{subequations}
\begin{align}
 N_{c} & \equiv 2\left( \frac{{m_{n}}^{*}k_{b}T}{2\pi\hbar^2}\right)^{3/2}\\
 N_{v}  &\equiv 2\left( \frac{{m_{p}}^{*}k_{b}T}{2\pi\hbar^2}\right)^{3/2}
\end{align}
\end{subequations}
The temperature dependencies of effective density of states is fairly simple:
\begin{subequations}
\begin{align}
 N_{c}\left(T\right) &= N_c \left( 300\Kelvin \right)\left( \frac{T}{300\Kelvin}\right)^{1.5}\\
 N_{v}\left(T\right) & = N_v \left( 300\Kelvin \right)\left( \frac{T}{300\Kelvin}\right)^{1.5}
\end{align}
\end{subequations}
\subsection{Bandgap}
The bandgap in GENIUS is expressed as follows:
\begin{subequations}
\begin{align}
 E_g (T) &= E_g (0) - \frac{\alpha T^{\rm 2} }{T + \beta} \nonumber\\
  &=E_g (300) + \alpha\left[ \frac{300^{\rm 2} }{300 + \beta} - \frac{T^2}{T + \beta} \right]
\end{align}
\end{subequations}
\subsection{Bandgap Narrowing due to Heavy Doping}When bandgap narrowing effects\label{Bandgap narrowing+Slotboom model}
due to heavy doping takes place \cite{Slotboom1977}, the band edge shifts:
\begin{equation}
\Delta E_g = \frac{E_{\rm bgn}}{2k_b T}\left[ \ln \frac{N_{\rm total}}{N_{\rm ref}} + \sqrt {\left(
        \ln \frac{N_{\rm total}}{N_{\rm ref}} \right)^2 + C_{\rm bgn}} \right].
\end{equation}
The intrinsic concentration should be modified:
\begin{equation}
n_{ie}=\sqrt{N_c N_v } \exp\left(-\frac{E_g}{2 k_b T} \right) \cdot \exp(\Delta E_g)
\end{equation}
Since the carrier current  Equation \eqref{eq:Equation:DDML1:DrivingField},

XXXXXXXXXXXXXXXXXXXXXXXXXX

%\at{p.}[eq:Equation:DDML1:DrivingField]
involves the energy level of
      conduction band $N_{c}$ and valence band $N_{v}$, the
      bandgap shift should be attributed to them. The bandgap narrowing is attributed half to the conduction band and
      another half to the valence band as default:
\begin{subequations}
\begin{align}
 E_c'  &=E_c-\frac{1}{2}\Delta E_g\\
 E_v'  &=E_v+\frac{1}{2}\Delta E_g
\end{align}
\end{subequations}
The parameters used in the default band structure model is listed in
Table \label{tab:Equation:Band:Default:Param}, %\at{p.}[tab:Equation:Band:Default:Param].
\par

\begin{table}
\centering
\begin{tabular}{llllll}
\toprule
Symbol &  Parameter &  Unit &  Silicon &  GaAs   \\
\midrule
$E_g(300)$ &  \cal{EG300} &  $\eVolt$ &  1.1241 &  1.424   \\
$\alpha$ &  \cal{EGALPH} &  $\eVolt\Per\Kelvin$ &  $2.73\times10^{-4}$ &  $5.405\times10^{-4}$   \\
$\beta$ &  \cal{EGBETA} &  $\Kelvin$ &  $0$ &  $204$   \\
$E_{\rm bgn}$ &  \cal{V0.BGN} &  $\eVolt$ &  $6.92\times10^{-3}$ &  0   \\
$N_{\rm ref}$ &  \cal{N0.BGN} &  $\Inverse\rm cm^3$ &  $1.30\times10^{17}$ &  $1\times10^{17}$   \\
$C_{\rm bgn}$ &  \cal{CON.BGN} &  - &  $0.5$ &  $0.5$   \\
$m_n$ &  \cal{ELECMASS} &  $m_0$ &  $1.0903$ &  $0.067$   \\
$m_p$ &  \cal{HOLEMASS} &  $m_0$ &  $1.1525$ &  $0.6415$   \\
$N_c(300)$ &  \cal{NC300} &  $\Inverse\rm cm^3$ &  $2.86\times10^{19}$ &  $4.7\times10^{17}$   \\
$N_v(300)$ &  \cal{NV300} &  $\Inverse\rm cm^3$ &  $3.10\times10^{19}$ &  $7.0\times10^{18}$   \\
\hline
\end{tabular}
\caption{Parameters of the Default band structure model}
\label{tab:Equation:Band:Default:Param}
\end{table}
\subsection{Band structure of compound semiconductors}
[[TODO]]
\par
\subsubsection{Band Structure of SiGe}
[[TODO]]
\par
\subsubsection{Band Structure of Tertiary Compound Semiconductor}
[[TODO]]
\par
\subsection{Bandgap}[[TODO]]
\par
\par
\subsection{Electron Affinity}[[TODO]]
\par
\par
\subsection{Effective Mass}[[TODO]]
\par
\par
\subsection{Density of States}[[TODO]]
\par
\par
\subsection{Schenk's Bandgap Narrowing Model}
[[TODO]] Equations of Schenk's model
\par
The Schenk's bandgap narrowing model is available for silicon, and can be loaded with the option
\cal{Schenk} in the \cal{PMI} command.
\par
\section{Carrier Recombination}
Three recombination mechanisms are considered in GENIUS at present, including Shockley-Read-Hall, Auger, and
      direct (or radiative) recombination. The total recombination is considered as the sum of all:
\begin{equation}
U = U_n = U_p = U_{\rm SRH} + U_{\rm dir} + U_{\rm Auger}
\end{equation}
where $U_{\rm SRH}$, $U_{\rm dir}$ and $U_{\rm Auger}$ are SRH recombination, direct recombination and Auger recombination,
      respectively.
\par
\subsection{SRH Recombination}Shockley-Read-Hall (SRH) recombination\label{SRH recombination}%\seeindex[]{Shockley-Read-Hall recombination}{SRH recombination}
rate is determined by the following formula:
\begin{equation}
U_{\rm SRH}=\dfrac{pn-{n_{ie}}^2}{\tau_p\left[n+n_{ie}\exp\left(\dfrac{\bf
        ETRAP}{kT_L}\right)\right]+\tau_n\left[p+n_{ie}\exp\left(\dfrac{ -{\bf ETRAP}}{kT_L}\right)\right]}
\end{equation}
where $\tau_n$ and $\tau_p$ are carrier life
      time\label{Carrier life-time}, which dependent on impurity concentration
\cite{Roulston1982}.
\begin{subequations}
\begin{align}
 \tau_n  &=\frac{{\bf TAUN0}}{1+N_{\rm total}/ {\bf NSRHN}}\\
 \tau_p  &=\frac{{\bf TAUP0}}{1+N_{\rm total}/ {\bf NSRHP}}
\end{align}
\end{subequations}
\subsection{Auger Recombination}The Auger recombination\label{Auger recombination}
is a three-carrier recombination process, involving either two electrons and one hole or two
        holes and one electron. This mechanism becomes important when carrier concentration is large.
\begin{equation}
U_{\rm Auger}={\bf AUGN} \left(pn^2-n{n_{ie}}^2 \right)+{\bf AUGP}(np^2-p{{n_{ie}}^2})
\end{equation}
where \cal{AUGN} and \cal{AUGP} are Auger coefficient for electrons and
      holes. The value of Auger recombination $U_{\rm Auger}$ can be negative some times,
      which refers to Auger generation.
\par
\subsection{Direct Recombination}The direct recombination\label{Direct recombination}
model expresses the recombination rate as a function of the carrier concentrations
$n$ and $p$, and the effective intrinsic density $n_{ie}$:
\begin{equation}
U_{\rm dir}={\bf DIRECT}(np-{n_{ie}}^2)
\end{equation}
The default value of the recombination parameters are listed in
 Table \ref{tab:Equation:Recomb:Param}, %p. 11:tab:Equation:Recomb:Param], \at{p.}[tab:Equation:Recomb:Param]:

\begin{table}
\centering
\begin{tabular}{llllll}
\toprule
 Parameter &  Unit &  Silicon &  GaAs &  Ge   \\
\midrule
 \cal{ETRAP} &  $\eVolt$ &  0 &  0 &  0   \\
 \cal{DIRECT} &  $\rm cm^3\rm s^{-1}$ &  1.1e-14 &  7.2e-10 &  6.41e-14   \\
 \cal{AUGN} &  $\text{cm}^6\rm s^{-1}$ &  1.1e-30 &  1e-30 &  1e-30   \\
 \cal{AUGP} &  $\text{cm}^6\rm s^{-1}$ &  0.3e-30 &  1e-29 &  1e-30   \\
 \cal{TAUN0} &  s &  1e-7 &  5e-9 &  1e-7   \\
 \cal{TAUP0} &  s &  1e-7 &  3e-6 &  1e-7   \\
 \cal{NSRHN} &  $\text{cm}^{-3}$ &  5e16 &  5e17 &  5e16   \\
 \cal{NSRHP} &  $\text{cm}^{-3}$ &  5e16 &  5e17 &  5e16  \\
\end{tabular}
\caption{Default values of recombination parameters}
\label{tab:Equation:Recomb:Param}
\end{table}



\subsection{Surface Recombination}At semiconductor-insulator interfaces, additional SRH recombination can be introduced. The surface
        recombination rate has the unit %$\text{cm}^{-2}\text{s}^{-1}$, and is calculated
        with
\begin{equation}
U_{\rm Surf}=\dfrac{pn-{n_{ie}}^2}{\dfrac{1}{\bf STAUN}\left(n+n_{ie}\right)+\dfrac{1}{\bf
        STAUP}\left(p+n_{ie}\right)}.
\end{equation}
The surface recombination velocities, \cal{STAUN} and \cal{STAUP}, have
      the unit of \cMeter\Per\Second, and the default value of 0.
\par
\section[sec:Equation:Mobility]{Mobility Models}
Carrier mobility is one of the most important parameters in the carrier transport model. The DD model
      itself, developed at early 1980s, is still being used today due to advanced mobility model enlarged its ability to
      sub-micron device.
\par
Mobility modeling is normally divided into: low field behavior, high field behavior and mobility in the
      (MOS) inversion layer.
\par
The low electric field behavior has carriers almost in equilibrium with the lattice. The low-field mobility
      is commonly denoted by the symbol $\mu_{n0}$, $\mu_{p0}$.
      The value of this mobility is dependent upon phonon and impurity scattering. Both of which act to decrease the low
      field mobility. Since scattering mechanism is depended on lattice temperature, the low-field mobility is also a
      function of lattice temperature.
\par
The high electric field behavior shows that the carrier mobility declines with electric field because the
      carriers that gain energy can take part in a wider range of scattering processes. The mean drift velocity no
      longer increases linearly with increasing electric field, but rises more slowly. Eventually, the velocity doesn't
      increase any more with increasing field but saturates at a constant velocity. This constant velocity is commonly
      denoted by the symbol $v_{sat}$. Impurity scattering is relatively insignificant for
      energetic carriers, and so $v_{sat}$ is primarily a function of the lattice
      temperature.
\par
Modeling carrier mobilities in inversion layers introduces additional complications. Carriers in inversion
      layers are subject to surface scattering, carrier-carrier scattering, velocity overshoot and quantum mechanical
      size quantization effects. These effects must be accounted for in order to perform accurate simulation of MOS
      devices. The transverse electric field is often used as a parameter that indicates the strength of inversion layer
      phenomena.
\par
It can be seen that some physical mechanisms such as velocity overshoot and quantum effect which can't be
      described by DD method at all, can be taken into account by comprehensive mobility model. The comprehensive
      mobility model extends the application range of DD method. However, when the EB method (which accounts for
      velocity overshoot) and QDD method (including quantum effect) are used, more calibrations are needed to existing
      mobility models.
\par
\subsection{Bulk Mobility Models}
The first family of mobility models were designed to model the carrier transport at low electric fields.
        They usually focus on the temperature and doping concentration dependence of the carrier mobilities. The
        surface-related or transverse E-field effects are {\em not}
included in these models. On the
        other hand, in GENIUS, these low-field mobilities models are coupled to a velocity saturation model to account
        for the carrier velocity saturation effect. This family of mobility models are suitable for bulk device, such as
        bipolar transistors.
\par
In brief, the low field carrier mobility is first computed, then a velocity saturation formula is applied
        to yield the corrected mobility value. Three choices are available for the low-field mobility calculation, each
        described in one of the following sub-sections. The choices of velocity saturation is described in the last
        sub-section.
\par
\subsubsection[sec:Equation:Mobility:Bulk:Analytic]{Analytic Mobility Model}
\label{mobility+Analytic model}In the GENIUS code, the Analytic Mobility model
\cite{Selberherr1984P} is the
          default low field mobility model for all the material. It is an concentration and temperature dependent
          empirical mobility model expressed as:
\begin{equation}
\mu_{0}=\mu_{\rm min}+\dfrac{\mu_{\rm max}\left(\dfrac{T}{300}\right)^\nu-\mu_{\rm
            min}}{1+\left(\dfrac{T}{300}\right)^\xi \left(\dfrac{N_{\rm total}}{N_{\rm ref}}\right)^\alpha}
\end{equation}
where $N_{\rm total}=N_A+N_D$ is the total impurity concentration.
\par
Default parameters for Si, GaAs and Ge are listed below:

\begin{table}
\begin{adjustwidth}{\dimexpr-\marginparwidth-\marginparsep}{}
\centering\begin{tabular}{lllllll}
\toprule
 Symbol &  Parameter &  Unit &  Si:n &  Si:p &  GaAs:n &  GaAs:p   \\
\hline
 $\mu_{\rm min}$ &  \cal{MUN.MIN} / \cal{MUP.MIN} &  $\text{cm}^2\text{V}^{-1}\text{s}^{-1}$ &  55.24 &  49.70 &  0.0 &  0.0   \\
 $\mu_{\rm max}$  & \cal{MUN.MAX} / \cal{MUP.MAX} &  $\text{cm}^2\text{V}^{-1}\text{s}^{-1}$ &  1429.23 &  479.37 &  8500.0 &  400.0   \\
 $\nu$  & \cal{NUN} / \cal{NUP} &  - &  -2.3 &  -2.2 &  -1.0 &  -2.1   \\
 $\xi$  & \cal{XIN} / \cal{XIP} &  - &  -3.8 &  -3.7 &  0.0 &  0.0   \\
 $\alpha$  & \cal{ALPHAN} / \cal{ALPHAP} &  - &  0.73 &  0.70 &  0.436 &  0.395   \\
 $N_{\rm ref}$ &  \cal{NREFN} / \cal{NREFP} &  $\text{cm}^{-3}$ &  1.072e17 &  1.606e17 &  1.69e17 &  2.75e17   \\
 \hline
\end{tabular}
\label{tab:Equation:Mobility:Analytic:Param}
\caption{Default parameter values of the analytic mobility model}
\end{adjustwidth}
\end{table}

In GENIUS, the analytic model is the simplest mobility model, and is available for a wide range of
          materials. For some materials, such as silicon, some more advanced mobility models are available.
\par
\par
\subsubsection[sec:Equation:Mobility:Bulk:Masetti]{Masetti Analytic Model}
\label{mobility+Masetii model}The doping-dependent low-field mobility model proposed by Masetti et
          al.\cite{Masetti1983} is an alternative to the default analytic model. The general expression
          for the low-field mobility is
\par

\begin{equation}
\mu_{\rm dop} = \mu_{\rm min1} \exp\left( -\frac{P_c} {N_{\rm tot}} \right) + \frac{\mu_{\rm
            const} - \mu_{\rm min2}}{1+\left( N_{\rm tot}/C_r \right)^\alpha } - \frac{\mu_1}{1 + \left( C_s/N_{\rm tot}
            \right)^\beta }
\end{equation}
where $N_{\rm tot}$ is the total doping concentration. The term
$\mu_{\rm const}$ is the temperature-dependent, phonon-limited mobility
\par

\begin{equation}
\mu_{\rm const} = \mu_{\rm max} \left( \frac{T}{300} \right)^{\zeta}
\end{equation}
where $T$ is the lattice temperature.
\par
The parameters of the Masetti model is listed in Table \label{tab:Equation:Mobility:Masetti:Param},
%\at{p.}[tab:Equation:Mobility:Masetti:Param].
%          The Masetti model is the default mobility model for the 4H-SiC material.
%\par



\begin{table}
\begin{adjustwidth}{\dimexpr-\marginparwidth-\marginparsep}{}
\centering
\begin{tabular}{lllll}
\toprule
 Symbol &  Parameter &  Unit &  4H-SiC:n &  4H-SiC:p  \\
\midrule
 $\mu_{\rm max}$ &  \cal{MUN.MAX} / \cal{MUP.MAX} &  $\text{cm}^2\text{V}^{-1}\text{s}^{-1}$ &  947.0 &  124.0  \\
 $\zeta$ &  \cal{MUN.ZETA} / \cal{MUP.ZETA} &  - &  1.962 &  1.424  \\
 $\mu_{\rm min1}$ &  \cal{MUN.MIN1} / \cal{MUP.MIN1} &  $\text{cm}^2\text{V}^{-1}\text{s}^{-1}$ &  0 &  15.9  \\
 $\mu_{\rm min2}$ &  \cal{MUN.MIN2} / \cal{MUP.MIN2} &  $\text{cm}^2\text{V}^{-1}\text{s}^{-1}$ &  0 &  15.9  \\
 $\mu_1$ &  \cal{MUN1} / \cal{MUP1} &  $\text{cm}^2\text{V}^{-1}\text{s}^{-1}$ &  0 &  0  \\
 $P_c$ &  \cal{PCN} / \cal{PCP} &  $\text{cm}^3$ &  0 &  0  \\
 $C_r$ &  \cal{CRN} / \cal{CRP} &  $\text{cm}^3$ &  $1.94\times 10^{17}$ &  $1.76\times 10^{19}$  \\
 $C_s$ &  \cal{CSN} / \cal{CSP} &  $\text{cm}^3$ &  0 &  0  \\
 $\alpha$ &  \cal{MUN.ALPHA} / \cal{MUP.ALPHA} &  - &  0.61 &  0.34  \\
 $\beta$ &  \cal{MUN.BETA} / \cal{MUP.BETA} &  - &  0 &  0  \\
\end{tabular}
\caption{Parameters of the Masetti mobility model}
\end{adjustwidth}
\label{tab:Equation:Mobility:Masetti:Param}
\end{table}

\subsubsection[sec:Equation:Mobility:Bulk:Philips]{Philips Mobility Model}
\label{mobility+Philips model}Another low field mobility model implemented into GENIUS is the Philips Unified Mobility model
\cite{Klaassen1992-1,Klaassen1992-2}. This model takes into account the
          distinct acceptor and donor scattering, carrier-carrier scattering and carrier screening, which is recommended
          for bipolar devices simulation.
\par
The electron mobility is described by the following expressions:
\begin{equation}
{\mu_{0,n}}^{-1} = {\mu_{{\rm Lattice},n}}^{-1} + {\mu _{D + A + p}}^{-1}
\end{equation}
where $\mu_{0,n}$ is the total low field electron mobilities, $\mu_{{\rm Lattice},n}$
is the electron mobilities due to lattice scattering, $\mu_{D + A + p}$
is the electron and hole mobilities due to donor (D), acceptor (A),
          screening (P) and carrier-carrier scattering.
\begin{align}
\mu_{{\rm Lattice},n}&=\mu_{\rm max} \left( \frac{T}{300} \right)^{-2.285}\\
\mu _{D + A + p} &= \mu_{1,n} \left( \frac{N_{{\rm sc,}n}}{N_{{\rm sc,eff,}n}} \right) \left(
            \frac{N_{\rm ref}}{N_{{\rm sc,}n} } \right)^\alpha + \mu_{2,n} \left( \frac{n + p}{N_{{\rm sc,eff,}n}}
            \right)
\end{align}
The parameters $\mu_{1,n}$ and $\mu_{2,n}$ are
          given as:
\begin{subequations}
\begin{align}
 \mu_{1,n}  &= \frac{ \mu_{\rm max }^2 }{ \mu_{\rm max } - \mu_{\rm min } }
            \left(\frac{T}{300} \right)^{3\alpha - 1.5}\\
 \mu_{2,n}  &= \frac{ \mu_{\rm max } \cdot \mu_{\rm min } }{ \mu_{\rm max } - \mu_{\rm min}
            } \left( \frac{300}{T} \right)^{1.5}
\end{align}
\end{subequations}
where ${N_{{\rm sc,}n} }$ and ${N_{{\rm
          sc,eff,}n}}$ is the impurity-carrier scattering concentration and effect impurity-carrier scattering
          concentration given by:
\begin{subequations}
\begin{align}
N_{{\rm sc,}n} &= {N_D}^* + N_A^* + p\\
N_{{\rm sc,eff,}n} &= {N_D}^* + {N_A}^* G\left( {P_n } \right) + \frac{p}{ F\left( {P_n } \right)
            }
\end{align}
\end{subequations}
where $N_D^*$ and $N_A^*$ take ultra-high doping
          effects into account and are defined by:
\begin{subequations}
\begin{align}
 {N_D}^* &= N_D \left( 1 + \dfrac{1}{C_D + \left( \dfrac{N_{D,{\rm ref}}}{N_D} \right)^2 }
            \right)\\
 {N_A}^* &= N_A \left( 1 + \dfrac{1}{C_A + \left( \dfrac{N_{A,{\rm ref}}}{N_A } \right)^2 }
            \right)
\end{align}
\end{subequations}
The screening factor functions $G\left( P_n \right)$ and $F\left( P_n \right)$
take the repulsive potential for acceptors and the finite mass of
          scattering holes into account.
\begin{equation}
G\left( P_n \right) = 1 - \frac{0.89233}{\left[ 0.41372 + P_n \left(
            \frac{m_0}{m_e}\frac{T}{300} \right)^{0.28227} \right]^{0.19778} } + \frac{0.005978}{\left[ P_n \left(
            \frac{m_e}{m_0}\frac{T}{300} \right)^{0.72169} \right]^{1.80618}}
\end{equation}
\begin{equation}
F\left( P_n \right) = \frac{0.7643{P_n}^{0.6478} + 2.2999 + 6.5502\frac{m_e}{m_h} }
            {{P_n}^{0.6478} + 2.3670 - 0.8552\frac{m_e}{m_h} }
\end{equation}
The $P_n$ parameter that takes screening effects into account is given
          by:
\begin{equation}
P_n = \left[ {\frac{ f_{cw} }{ N_{\rm sc,ref} \cdot {N_{{\rm sc,}n}}^{-2/3} } + \dfrac{ f_{BH}
            }{ \dfrac{ N_{\rm c,ref} } {n + p}\left( \dfrac{m_e}{m_0} \right)}} \right]^{-1} \left( \frac{T}{300}
            \right)^2
\end{equation}
Similar expressions hold for holes. The default parameters for Philips model are listed in Table
%\inTab[tab:Equation:Mobility:Philips:Param], \at{p.}[tab:Equation:Mobility:Philips:Param]:
%\par




\begin{table}
\begin{tabular}{lllll}
\toprule
 Symbol &  Parameter &  Unit &  Si:n &  Si:p  \\
\midrule
 $\mu_{\rm min}$ &  \cal{MMNN.UM} / \cal{MMNP.UM} &  $\text{cm}^2\text{V}^{-1}\text{s}^{-1}$ &  55.24 &  49.70  \\
 $\mu_{\rm max}$ &  \cal{MMXN.UM} / \cal{MMXP.UM} &  $\text{cm}^2\text{V}^{-1}\text{s}^{-1}$ &  1417.0 &  470.5  \\
 $\alpha$ &  \cal{ALPN.UM} / \cal{ALPP.UM} &  - &  0.68 &  0.719  \\
 $N_{\rm ref}$ &  \cal{NRFN.UM} / \cal{NRFP.UM} &  $\text{cm}^{-3}$ &  9.68e16 &  2.23e17  \\
 $C_D$ &  \cal{CRFD.UM} &  - &  0.21 &  0.21  \\
 $C_A$ &  \cal{CRFA.UM} &  - &  0.5 &  0.5  \\
 $N_{\rm D,ref}$ &  \cal{NRFD.UM} &  $\text{cm}^{-3}$ &  4.0e20 &  4.0e20  \\
 $N_{\rm A,ref}$ &  \cal{NRFA.UM} &  $\text{cm}^{-3}$ &  7.2e20 &  7.2e20  \\
 $m_e$ &  \cal{me\_over\_m0} &  $m_0$ &  1.0 &  -  \\
 $m_h$ &  \cal{mh\_over\_m0} &  $m_0$ &  - &  1.258  \\
 $f_{cw}$ &  - &  2.459 &  2.459  \\
 $f_{BH}$ &  - &  3.828 &  3.828  \\
 $N_{\rm sc, ref}$ &  \cal{NSC.REF} &  $\text{cm}^{-2}$ &  3.97e13 &  3.97e13 \\
 $N_{\rm c,ref}$ &  \cal{CAR.REF} &  $\text{cm}^{-3}$ &  1.36e20 &  1.36e20 \\
 \hline
\end{tabular}
\caption{Default values of Philips mobility model parameters}
\label{tab:Equation:Mobility:Philips:Param}
\end{table}

In the actual code, Philips model is corrected by Caughey-Thomas expression for taking high field
          velocity saturation effects into account. This model can be loaded by
\cal{Philips} keyword
          in the \cal{PMI} statements.
\par
\subsubsection[sec:Equation:Mobility:Bulk:VSat]{Velocity Saturation}
\label{velocity saturation}
\par
\subsection{Silicon-like materials}\label{velocity saturation+Caughey-Thomas model}For silicon-like materials, the Caughey-Thomas expression
\cite{Caughey1967}, is
            used:
\begin{equation}
\mu = \dfrac{\mu _{0} }{\left[ 1 + \left( \dfrac{\mu _{0} E_{\parallel} }{v_{\rm sat} }
            \right)^\beta \right]^{1/\beta} }
\end{equation}
where $E_{\parallel}$ is the electric field parallel to current flow.
$v_{\rm sat}$ is the saturation velocities for electrons or holes. They are
          computed by default from the expression:
\begin{equation}
v_{\rm sat} (T) = \frac{ v_{\rm sat0} }{ 1 + \alpha \cdot \exp \left( \frac{T}{600} \right)
            }
\end{equation}



\chapter{The Second Chapter Style}
\end{document} 